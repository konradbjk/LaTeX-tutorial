\documentclass[11pt]{article}

\usepackage{graphicx}
\usepackage{hyperref}
\usepackage{float}

\begin{document}


\title{LateX document}
\author{Konrad Bujak}
\date{\today} 
\maketitle

\tableofcontents

This gives automatic date

\section{Section 1}
	\subsection{greek letters}

$$\pi$$
$$\alpha$$
$$A=\pi r^2$$

	\subsection{trig functions}
$$y = \sin{x}$$
$$y = sin(x)$$

	\subsection{logaritms}
$$\log_5{x}$$
$$\ln{x}$$

	\subsection{fractions}
$$\frac{1}{2}$$
There is $\frac{2}{3}$ of the water in glass. You can also make this fraction bigger using displaystyle $\displaystyle{\frac{x}{x^2-12x+3}}$ as you can see.

	\subsection{square roots}
$$\sqrt{2}$$
$$\sqrt[3]{8} = 2$$
$$\sqrt[3]{3x+123+\sin{x}}$$
$$\sqrt{\frac{1}{\sqrt{x}}}$$

\section{Section 2} \label{sec:section2} 


	\subsection{brackets}
$$(x+1)$$
$$3[2+(x+1)]$$
$$3x\left(\frac{2}{4}\right)$$
$$3x\left[2+\frac{2}{3}\right]$$

curly brackets:
$$\{a,b,c\}$$
$$3x\left\{2+\frac{2}{3}\right\}$$
$$\left\{x^2\right.$$

	\subsection{Signs}	
	
Dollar symbol: $\$12.55$

Absolute value:
$$|x|$$
$$\left|\frac{x}{x-1}\right|$$
$$\left. \frac{dy}{dx} \right|_{x=1}$$

	\subsection{Tables}

\begin{tabular}{c|ccccc}

$x$ & 1 & 2 & 3 & 4 & 5 \\ \hline 
$f(x)$ & 10 & 12 & 13 & 14 & 15 \\

\end{tabular}

\begin{eqnarray}
5x^2-9&=& x+3\\
4x^2&=& 12\\
x^3&=& 3\\
x&\approx&\pm1.732
\end{eqnarray}
Hiding equation number:
\begin{eqnarray*}
4x^2&=& 12\\
x^3&=& 3
\end{eqnarray*}

\begin{eqnarray}
5x^2-9&=& x+3
\end{eqnarray}

	\subsection{Lists}

		\subsubsection{Enumeraterd Lists}
\begin{enumerate}
\item penicl
\item calculator
\item ruler
\item paper
\item notebook
	\begin{enumerate}
	\item red notebook
		\begin{enumerate}
		\item pages 1-3
		\begin{enumerate}
		\item page 1
		\end{enumerate}
		\end{enumerate}
	\item blue notebook

	\end{enumerate}
\end{enumerate}

		\subsubsection{Itemized list}

\begin{itemize}
\item pencl
\item calculator
\item ruler
\item paper
\item notebook
	\begin{itemize}
	\item red notebook
		\begin{itemize}
		\item pages 1-3
			\begin{itemize}
			\item page 1
			\end{itemize}
		\end{itemize}
	\item blue notebook

	\end{itemize}

\end{itemize}

\begin{enumerate}
\item[item1]: penicl
\item[item2]: calculator
\item[item3] ruler
\item paper
\item notebook
\end{enumerate}

\section{Text formating}

You can make text \textit{italic} or make it \textbf{bold face} or \textsc{small caps}. Also there is \texttt{typewriter}, for example for urls \texttt{https://google.com}

\begin{center}
This text is center
\end{center}

\begin{flushleft}
This is left-justified
\end{flushleft}
\begin{flushright}
This is right-justified
\end{flushright}

\section{Graphic}
To include graphics, please remember about inserting package \textbf{graphicx}.
\begin{figure}[H]
	\centering
	\includegraphics[width=\textwidth,height=\textheight,keepaspectratio]{img/image1.png}
	\caption{Nike shoes customization }
	\label{fig:nike_shoes}
\end{figure}

But remember that images must be as .png, .jpg, .gif or .pdf

\section{Hyper Links}


	\subsection{\textbf{\textbackslash hyperref}}

We use \hyperref[nike_shoes]{figure \ref*{fig:nike_shoes}} to show the promotions and customization of nike shoes. We use asterisk here to avoid nested hyperlinks. We receive the number of figure that ws labeled.

Also using labels we can directly link to specific point in document.
This \hyperref[fig:nike_shoes]{picture} shows customization of shoes.

In \hyperref[sec:section2]{Section 2} we have shown how to use brackets.

	\subsection{\textbf{\textbackslash url}}
This function just makes clicakble urls

\url{https://en.wikibooks.org/wiki/LaTeX/Hyperlinks}

	\subsection{\textbf{\textbackslash href}}

This function shows our description of url but when clicked on it, the web browser points to given url.

\href{https://www.wikibooks.org}{Wikibooks home}

\end{document}
